\documentclass{article}

\usepackage{amsmath,amssymb,scalerel,amsthm}
\usepackage{derivative}
\usepackage{mathtools}
\usepackage{pgfplots}

\usepackage{tikz}
\usepackage{color}
\pgfplotsset{compat=1.8}
\usetikzlibrary{arrows}

\title{General Physics}
\author{Alessio Esposito}
\begin{document}
    \maketitle
    \section{Dynamics}
        \subsection{The three laws of Dynamics}
            \begin{itemize}
                \item A body continues moving if no force comes to act on it. Same for a body that is not moving
                \item In a system of coordinates comes the following equivalence: 
                    \begin{equation*}
                        \vec{F} = m \vec{a}             
                    \end{equation*}
                \item If a body $A$ puts a force on a body $B$ then the body $B$ also puts a force on the body $A$.
            \end{itemize}
            \section{Angular Momentum}
                We define the Angular Momentum as follows:
                \begin{equation*}
                    L = \vec{r_v} \times \vec{P}
                \end{equation*}
                Where $\vec{P}$ is equal to $m\vec{v}$ and $\vec{v}$ is the velocity of the point, so in a more general notation we have:
                \begin{equation*}
                    \frac{d\vec{P}}{dt} = \frac{d}{dt}(m\vec{v}) = m \frac{d\vec{v}}{dt} = m \vec{a} = \vec{F} 
                \end{equation*}
                $\vec{r_v}$ instead is the vector that starts froma a chosen pole and goes to the moving point. \\ 
                With that said we have $L = \vec{r_v} \times m\vec{v}$.
                
                So the angular Momentum is equal to the ortogonal vector generate by the vectorial prodouct of $\vec{r_v}$ and $m\vec{v}$, basically $L = r_v mv \sin{\theta}$, where $\theta$ is the angle generated by the two vectors.
            \section{Differential equations}
                Lets suppose that the $\vec{F} = m \vec{a}$ its dependand by the following variables. 
                \begin{equation*}
                    \vec{F} = (\vec{r(t)}, \frac{d\vec{r}}{dt}, t) = m \frac{d^2 \vec{r}}{d t^2}
                \end{equation*}
                this relation is called differential equation of the second order, because there is a derivative of the second order.
                \subsection*{Example}
                    If we consider the Hooke's law, one has the following differential equation:
                    \begin{equation*}
                        m\ddot{x} = - K x 
                    \end{equation*}
                    Where $\ddot{x}$ is the second derivative of $x = x(t)$ and $K$ is the Hooke's constant.
                    more precisely if we call $\omega^2 = \frac{K}{m}$ we have:
                    \begin{equation*}
                        \ddot{x} = -\omega^2 x
                    \end{equation*}
                    Having that said, if we want to resolve the current equation we have to find the initial conditions for the "Cauchy's problem" to be applied. \\
                    Lets put $\dot{x}(0) = 0$ and lets also suppose $x(0) = 0$. \\
                    the Cauchy's problem can be solved so if we have $\omega^2 = 1$ we will have:
                    \begin{equation*}
                        \ddot{x} = -x 
                    \end{equation*}

                    so the following equivalent solutions will be:
                    \begin{equation*}
                        \dot{x} = \omega \cos(\omega t)
                    \end{equation*}
                    and 
                    \begin{equation*}
                        \dot{x} = -\omega^2 \sin(\omega t)
                    \end{equation*}
                    With $x(t) = \sin(\omega t)$ as the general solution.

            % \subsubsection{Example}
            %     \begin{tikzpicture}
            %         \begin{axis}[
            %             view={35}{25},
            %             axis lines=center,
            %             width=7cm,height=7cm,
            %             xlabel={$x$},ylabel={$y$},zlabel={$z$},
            %             xmin = 0, xmax = 5, ymin = 0, ymax = 5, zmin = 0, zmax = 5,
            %           ]
                      
            %           \pgfplotsset{ticks=none}
            %           % plot dashed lines to axes
            %           \addplot3 [no marks,densely dashed] coordinates {(0,0,3) (3,4,2)};
            %           \addplot3 [no marks,densely dashed] coordinates {(4,0,0) (3,4,2)};
            %           \addplot3 [no marks,densely dashed] coordinates {(0,4,0) (3,4,2)};
            %           % label points
            %           \node [above right] at (axis cs:3,4,2) {$P (x,y,z)$};
                    
            %         \end{axis}

            %     \end{tikzpicture}

            %     Lets suppose to have two inertial systems and a body with the following laws of time:
            %     \begin{equation*}
            %         \begin{cases}
            %             x(t) = x_0 + x' \\
            %             y(t) = y_0 + y' \\
            %             z(t) = z_0 + z'
            %         \end{cases}
            %     \end{equation*}
            %     Where $x_0, y_0, z_0$ are the coordinates of the first inertial system $S$.
            %     Lets suppose that  $S$ translates with the $y$ axes the velocity will be:
            %     \[
            %         \begin{cases}
            %             V_x(t) = v'_x(t) \\
            %             V_x(t) = v_0 + v'_y(t) \\
            %             V_z(t) = v'_x(t)    
            %         \end{cases}
            %     \]

\end{document}