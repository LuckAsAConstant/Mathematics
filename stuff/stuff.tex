\documentclass{article}

\usepackage{amsmath,amssymb,scalerel,amsthm}
\usepackage{derivative}
\usepackage{mathtools}
\usepackage{color}

\begin{document}
    \section*{Monomi}
        \begin{itemize}
            \item $ax$
            \item $axy$
            \item $ax^2$
            \item $axy^3$
            \item $axyz$
        \end{itemize}
    esempio con i numeri: $5xy$

    \section*{Polinomi}
        \begin{itemize}
            \item $ax^2 + bx + c$
            \item $ax^3 + bx^2 + cx + d$
            \item $axy^2 - c$
            \item $ax^3 + cx - d$
            \item $axyz^3 + bxy + cz$
        \end{itemize}
        Il mononio con il grado piu alto determina il grado del polinomio \\
        esempio: $5xyz^3 + 6y^3 - 7z$ e' di grado $3$
    \section*{Operazioni}
        \subsubsection*{Addizione}
            esempio: $5xy + 3x$ deve essere sommato a $4xy - 2x$. Procedimento: \\
            la somma e': $5xy + 3x + 4xy - 2x$ ma alcuni monomi sono simili, quindi sommiamoli tra di loro:
            $9xy - x$
        \subsubsection*{Sottrazione}
            e' la stessa cosa tranne che per una accortezza. Prendiamo gli stessi polinomi di prima e sottraiamoli: \\
            avremo $5xy + 3x - (4xy - 2x) = 5xy + 3x - 4xy + 2x = xy + 5x$
            % consideriamo i monomi simili ad esempio $5xy$ e $4xy$
        \subsubsection*{Moltiplicazione}
            Abbiamo due polinomi $x - 1$ e $x + 2$ la loro Moltiplicazione e' fatta moltiplicando componente per componente:
            $x^2 +2x - x  -2 = x^2-x -2$
            \newpage
        \section*{Esercizi}
        $(2x -1)(x^2 -3x +2) = 2x^3 - 6x^2 + 4x -x^2 + 3x - 2 = 2x^3 -6x^2 -x^2 +3x + 4x - 2 = 2x^3 - 7x^2 + 7x -2.$
\end{document}