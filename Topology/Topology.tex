\documentclass{article}

\usepackage{amsmath,amssymb,scalerel,amsthm}
\usepackage{derivative}
\usepackage{mathtools}
\usepackage{color}


\title{Topology}
\author{Alessio Esposito}

\newtheorem{proposition}{Proposition}
\newtheorem{definition}{Definition}
\newtheorem{example}{Example}
\newtheorem{theorem}{Theorem}
\newtheorem{lemma}{Lemma}

\begin{document}
\maketitle
\begin{definition}
    let $\mathsf{X}$ be a not empty set and $\mathcal{T}$ a collection of its subsets. $\mathcal{T}$ is said topology on $\mathsf{X}$ if it has the following propeties:
    \begin{itemize}
        \item $\varnothing, \mathsf{X} \in \mathcal{T}$
        \item the union of a whatever family of elements of $\mathcal{T}$ is in $\mathcal{T}$. In simbols: $\forall \{A_i \}_{i \in I}, \ A_i \in \mathcal{T}, \ \bigcup_{i \in I}A_i \in \mathcal{T}$  
        \item the interection of elements of $\mathcal{T}$ is in $\mathcal{T}$: \ $\forall A_1, A_2 \in \mathcal{T}, \ A_1 \cap A_2 \in \mathcal{T} $ 
    \end{itemize}
Then $(\mathsf{X},\mathcal{T})$ its said to be topological space and $\mathsf{X}$ is called support. 
\begin{example}
    Let $\mathsf{X} = \mathbb{R}^2$ with the euclidean distance and $\forall c \in \mathbb{R}^2$ and $\forall r > 0$ the set $B_r(c) = \{ x \in \mathbb{R}^2 : d(x,c) < r \}$ its said open spherical neighbourhood of center $c$ and radius $r$.
    Let $\mathcal{T}$ be the totality of all the possible unions of open spherical neighbourhoods. $\mathcal{T}$ is a topology.

\end{example}
\end{definition}
\begin{definition}
    $(\mathsf{X}_1, \mathcal{T}_1), \ (\mathsf{X}_2, \mathcal{T}_2)$ topological spaces with $\mathsf{X}_1 \cap \mathsf{X}_2 = \varnothing.$ 
     \\ let $X = \mathsf{X}_1 \cup \mathsf{X}_2, \mathcal{T} = \{ A_1 \cup A_2 \ | \ A_i \in \mathcal{T}_i \}.$
\end{definition}
    \begin{lemma}
        $(\mathsf{X}, \mathcal{T})$ is a topological space.
    \end{lemma}    
    \begin{proof}
        Lets verify the axioms.
        \begin{enumerate}
            \item $ \varnothing \in \mathcal{T}_1, \ \varnothing \in \mathcal{T}_2 \Rightarrow \varnothing = \varnothing \cup \varnothing \\ \mathsf{X}_i \in \mathcal{T}_i \Rightarrow \mathsf{X} = \mathsf{X}_1 \cup \mathsf{X}_2 \in \mathcal{T}$
            \item $\{A\}_{i\in I} \in \mathcal{T} \Rightarrow A_i = A_{1,i} \cup A_{2,i} \\ \bigcup_{i \in I}A_i = \bigcup_{i \in I}(A_{1,i} \cup A_{2,i}) = \bigcup_{i \in I}(A_{1,i}) \cup \bigcup_{i \in I}(A_{2,i}) \in \mathcal{T}_1 \cup \mathcal{T}_2$
            \item $ A, A' \in \mathcal{T} \Rightarrow A = A_1 \cup A_2, A' = A_1' \cup A_2' \\ A \cap A' = (A_1 \cup A_2) \cap (A_1' \cup A_2') = (A_1 \cap A_1') \cup (A_2 \cap A_2') \in \mathcal{T}  $ 
        \end{enumerate}

    \end{proof}
    \begin{definition}
        $\mathcal{T}$ is defined as a sum of Topologies $\mathcal{T}_1$ and $\mathcal{T}_2$. \\ We notice that $\forall A_i \in \mathcal{T}_i, \ A_i \cup \varnothing \in \mathcal{T}$ so $\mathcal{T}$ contains $\mathcal{T}_1$ and $\mathcal{T}_2.$  

    \end{definition}
        
    \begin{definition}
        $(\mathsf{X}, \mathcal{T})$ topological space $\mathsf{Y} \subseteq \mathsf{X}, \ \mathsf{Y} \neq \varnothing \\ \mathcal{T}_{/Y} = \{ A\cap \mathsf{Y} \ | \ A \in \mathcal{T} \}$ prove that $\mathcal{T}_{/Y}$ is a topology defined as Inducted topology on $Y$
    \end{definition}
    \begin{definition}
        $(\mathsf{X}, \mathcal{T})$ topological space, $\mathsf{X}$ verifies the second axiom of numerability if posseses a finite base or numerable, in that case $(\mathsf{X}, \mathcal{T})$ is said $\mathcal{N}_2$
    \end{definition}
    \begin{proposition}
        Let $\mathbb{R}$ be gifted by the topology with a base of the following type: 
        \begin{equation*}
            [a,b], \ a < b    
        \end{equation*}
        Then $(\mathbb{R}, \mathcal{T})$ is not $\mathcal{N}_2$
    \end{proposition}
    \begin{proof}
        Let $\mathcal{B}$ a base for $\mathcal{T}$. Let $a > 0 \in \mathbb{R}$ then $\forall x  \in \mathbb{R}$, there exists $B_x \in \mathcal{B}$ with $x \in B_x \subseteq [x, x + a]$.
        If $y \in \mathbb{R}$ with $y > x$ then $x \notin [y, y + a]$ so $x \notin B_y$.
        The application $x\in \mathbb{R} \longmapsto B_x \in \mathcal{B}$  is injective so $ \mathcal{B}$ has the continuum order.    
    \end{proof}
    \section*{Continous functions and homeomorphisms}
    \begin{definition}
        Let $(\mathsf{X}, \mathcal{T}_x)$, $(\mathsf{Y}, \mathcal{T}_y)$ be topological spaces, $\varOmega : \mathsf{X} \rightarrow \mathsf{Y}$ is continous in $a\in \mathsf{X}$ if $\forall I $ neighbourhood of $\varOmega(a)$, $\exists K$ neighbourhood of $a$ s.t. $\varOmega(K) \subseteq I.$  
    \end{definition}
    We'll say that a function is continous if it is continous in every point.
    \begin{proposition}
        Let $\varOmega:(\mathsf{X}, \mathcal{T}_x) \rightarrow (\mathsf{Y}, \mathcal{T}_y)$ so then the following affirmations are equivalent:
        \begin{itemize}
            \item[i.] $\varOmega$ is continous.
            \item[ii.] $\forall A \in \mathcal{T}_y,\ \varOmega^{-1}(A) \in \mathcal{T}_x.$
            \item[iii.] $\forall c \in \mathcal{C}(\mathsf{Y}), \ \varOmega^{-1}(c) \in \mathcal{C}(\mathsf{X}).$
            \item[iv.] The counterimages of opens under a selected base of $\mathsf{Y}$ are opens of $\mathsf{X}.$
            \item[v.] $\forall b = \varOmega(a) \in Im\varOmega = \varOmega(\mathsf{X})$ the counterimage  of every neighbourhood $K'$ of $b$ is a neighbourhood of $a$      
        \end{itemize}
        \begin{proof}
            $i. \Rightarrow ii.$ \ Let $A \in \mathcal{T}_y$ we have to prove that $\forall a \in \varOmega^{-1}(A)$ exists a neighbourhood of $a$ contained in $\varOmega^{-1}(A).$ For $a \in \varOmega^{-1}(A)$ one has $\varOmega(a) \in \varOmega(\varOmega^{-1}(A)) = A.\ A$ is open and is a neighbourhood of $\varOmega(a)$ so if $\varOmega$ is continous there exists a neighbourhood $K$ of $a$ s.t. 
            $\varOmega(K) \subseteq A$ hence $K \subseteq \varOmega^{-1}(A)$ and $a \in K.$ \\
            $ii. \Rightarrow i.$ Let $a \in \mathsf{X}, \ I$ neighbourhood of $\varOmega(a)$ let $A \in \mathcal{T}_y$ s.t. $\varOmega(a) \in A \subseteq K$ for the $ii.$ $\varOmega^{-1}(A)$ is open and is a neighbourhood of $a$ s.t. its image contains $\varOmega(a)$ and $A$ is contained in $I$.
        \end{proof}
    \end{proposition}
    
\end{document}