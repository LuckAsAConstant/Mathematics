\documentclass{article}

\usepackage{amsmath,amssymb,scalerel,amsthm}
\usepackage{derivative}
\usepackage{mathtools}
\usepackage{pgfplots}

\usepackage{tikz}
\usepackage{color}
\pgfplotsset{compat=1.8}
\usetikzlibrary{arrows}

\title{Rational Mechanics}
\author{Alessio Esposito}

\newtheorem{proposition}{Proposition}
\newtheorem{definition}{Definition}
\newtheorem{theorem}{Theorem}

\begin{document}
\maketitle
    \section*{Central Movements}
        Let $\vec{a} = f(\left\lVert P - O \right\rVert) (P-O)$, where we suppose to have a function $f : A \subseteq \mathbb{R} \rightarrow \mathbb{R}$ that every time the acceleration is proportional to the movement $P - O$ we will have a central movement.\\
        
        \subsection*{Example}
            \begin{tikzpicture}    
                \begin{axis}[
                        view = {35}{25},
                        axis lines=center,
                        width=7cm, height=7cm,
                        xlabel={$x$}, ylabel={$y$},
                        xmin = 0, xmax = 5, ymin = 0, ymax = 5]

                        \pgfplotsset{ticks=none}
                        \addplot [no marks,densely dashed] coordinates {(0,3) (3,3)};
                        \addplot [no marks,densely dashed] coordinates {(3,0) (3,3)};
                        \addplot [smooth] coordinates {(0,0) (3,3)};

                        \node [above right] at (axis cs:0,1) {$i$};
                        \node [above right] at (axis cs:1,0) {$j$};
                        \node [above right] at (axis cs:3,3) {$P$};

                    \end{axis}
            \end{tikzpicture}  

        We define: $\vec{a} = - \omega ^2 (P - O)$ with $\omega^2 > 0$ where $P-O$ is the movement vector in the central movement.
        So we can express the acceleration with the following equation:
        \begin{equation*}
            \vec{a} = \ddot{x}i + \ddot{y}j
        \end{equation*}
\end{document}