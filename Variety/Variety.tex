\documentclass{article}

\usepackage{amsmath,amssymb,scalerel,amsthm}
\usepackage{derivative}
\usepackage{mathtools}
\usepackage{color}


\title{Spiegazioni}
\author{Alessio Esposito}

\newtheorem{proposition}{Proposition}
\newtheorem{definition}{Definition}
\newtheorem{theorem}{Theorem}

\begin{document}
\maketitle
    \section*{Numeri complessi}
        L'insieme dei numeri complessi è così definito:
        \begin{equation*}
            \mathbb{C} := \{ \big(\begin{smallmatrix} a & b \\ -b & a \end{smallmatrix} \big) : a,b \in \mathbb{R} \}
        \end{equation*}
        Da notare come la coppia $(a,b)$ identifica univocamente la matrice $\big(\begin{smallmatrix} a & b \\ -b & a \end{smallmatrix} \big)$, perciò se definiamo $i = (0,1)$ e $1 = (1, 0)$ possiamo concludere che qualsiasi numero reale $a$ può essere nella forma $(a,0)$. \\
        Per quanto scritto sopra, possiamo definire $\mathbb{C}$ come segue:
        \begin{equation*}
            \mathbb{C} := \{ (a,b) : a,b \in \mathbb{R} \}
        \end{equation*} 
        Questa definizione porta al seguente lemma:
       \subsection*{Lemma}
        L'applicazione $\varsigma : \mathbb{R}^2 \rightarrow \mathbb{C}$ è un isomorfismo
        \begin{proof}
            La dimostrazione è banale basta considerare la definizione di $\mathbb{R}^2$.    
        \end{proof}
    Bisogna precisare però il fatto che $\mathbb{C}$ è isomorfo a $\mathbb{R}^2$ solo se li si considera come spazi vettoriali, infatti per costruire un isomorfismo tra algebre bisogna definire una nuova struttura come segue:
    \begin{definition}
        Il prodotto tra vettori è dato dall'operatore binario: 
        \begin{equation*}
            \xi : \mathbb{R}^2 \times \mathbb{R}^2 \rightarrow \mathbb{R}^2
        \end{equation*}
        tale che ad ogni coppia $((a,b),(c,d))$ associa il vettore $(ac - bd, bc + ad).$

    \end{definition}

    Consideriamo il numero complesso $z \in \mathbb{C}$ con $z = (a,b)$, con l'operazione appena definita possiamo perciò costruirlo come segue:
    \begin{equation*}
        a + ib = (a,0) + (0,1)(b,0) = (a, 0) + (0,b) = (a,b) 
    \end{equation*}
        Dotando perciò lo spazio vettoriale di questo prodotto si ottiene la capacità di dividere uno scalare per un vettore di $\mathbb{R}^2$. \\
        Infatti: 
        \begin{equation*}
            \frac{\alpha }{(\beta ,\gamma)} = \alpha (\delta ,\varepsilon) = (\alpha, 0)(\delta ,\varepsilon) = (\alpha \delta, \alpha \varepsilon )
        \end{equation*}

        Dove $\alpha,\beta,\gamma \in \mathbb{R}^2$ e $(\delta ,\varepsilon) = (\beta, \gamma)^{-1}$. \\
        Tale inverso esiste perchè definendo così il prodotto abbiamo reso $\mathbb{R}^2$ un campo.

        la definizione di derivata torna ad avere senso.

    \section*{Topologia dei cerchi e dei rettagoli}
        Siano $(\mathbb{R}^n, \mathcal{T}_1)$ e $(\mathbb{R}^n, \mathcal{T}_2)$ due spazi topologici, con $\mathcal{T}_1$ la totalità dei cerchi di centro $c \in \mathbb{R}^n$ e di raggio $r \in \mathbb{R}_+$
        e $\mathcal{T}_2$ la totalità dei rettangoli del tipo: \\ $(x_1,x_2)\times \dots \times (x_{n-1},x_n)$ dove $x_1,\dots ,x_n \in \mathbb{R}$. \\ 
        segue definizione: 

        \begin{proposition}
            Sia $\mathcal{B} = \{ B_i \}_{i\in I}$ una base dello spazio topologico $\mathcal{T}_1$ con \\ $B_i = \{ B_\frac{1}{n}(c) : n \in \mathbb{N}, c \in \mathbb{R}^n \}$, Allora vale l'uguaglianza:
            \begin{equation*}
                \bigcup_{i \in I} B_i = \bigcup_{j\in J} R_j
            \end{equation*}
            dove $R_j$ è la totalità dei rettangoli della topologia $\mathcal{T}_2$ \footnote{Ho preso la totalità di tutti i rettagoli come base del secondo spazio per motivi di praticità, la definizione di base viene comunque rispettata.}.
        \end{proposition}
        \begin{proof}
            ($\supseteq $) $\mathcal{OSS:}$ cominciamo con l'osservare che la diagonale di un rettangolo si ricava in questo modo: $d = \sqrt{h^2 + b^2}$ dove $b$ e $h$ sono rispettivamente base e altezza.
            Possiamo applicare questo concetto anche a degli intervalli, infatti se un rettangolo in $\mathbb{R}^2$ è definito come $(a,b) \times (c,d)$ basta prendere come base $\left\lvert a - b \right\rvert$ e come altezza $\left\lvert c - d \right\rvert$.
            Sia perciò $R$ un rettangolo definito in questo modo: $R = (x_1,x_2)\times \dots \times (x_{n-1},x_n)$ tale che, fissato un cerchio della base $\mathcal{B}$ con raggio $\frac{1}{n}$ rispetti la seguente condizione:
            \begin{equation*}
                \sqrt{{\left\lvert x_1 - x_2 \right\rvert}^2 + \dots + {\left\lvert x_{n-1} - x_n \right\rvert}^2} < \frac{2}{n}
            \end{equation*}
            Dove $\frac{2}{n}$ è il diametro del cerchio. \\ 
            Con questa condizione possiamo concludere che ogni cerchio contiene un rettangolo. Si può concludere che $\bigcup_{i \in I} B_i \supseteq \bigcup_{j\in J} R_j.$ \\
            ($\subseteq$) Analogamente al caso precedente possiamo usare concetti geometrici del genere per provare la tesi.
            Si noti che possiamo prendere un qualsiasi rettangolo tale che venga rispettata questa condizione: 
            \begin{equation*}
                \left\lvert x_i - x_{i + 1} \right\rvert \geq \frac{2}{n} \qquad \forall i = 1,\dots , n - 1
            \end{equation*}
            con queste considerazioni si può infine affermare che $\bigcup_{i \in I} B_i \subseteq \bigcup_{j\in J} R_j$ e le topologie coincidono, quindi si ha l'asserto.
         \end{proof}
\end{document}