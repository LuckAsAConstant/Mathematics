\documentclass{article}

\usepackage{amsmath,amssymb,scalerel,amsthm}
\usepackage{derivative}
\usepackage{mathtools}
\usepackage{color}


\title{Spiegazioni}
\author{Alessio Esposito}

\newtheorem{proposition}{Proposition}
\newtheorem{definition}{Definition}
\newtheorem{theorem}{Theorem}

\begin{document}
\maketitle
    \section*{Numeri complessi}
        l'insieme dei numeri complessi è così definito:
        \begin{equation*}
            \mathbb{C} := \{ \big(\begin{smallmatrix} a & b \\ -b & a \end{smallmatrix} \big) : a,b \in \mathbb{R} \}
        \end{equation*}
        Da notare come la coppia $(a,b)$ identifica univocamente la matrice $\big(\begin{smallmatrix} a & b \\ -b & a \end{smallmatrix} \big)$, perciò se definiamo $i = (0,1)$ e $1 = (1, 0)$ possiamo concludere che qualsiasi numero reale $a$ può essere nella forma $(a,0)$. \\
        Per quanto scritto sopra, possiamo definire $\mathbb{C}$ come segue:
        \begin{equation*}
            \mathbb{C} := \{ (a,b) : a,b \in \mathbb{R} \}
        \end{equation*} 
        Questa definizione porta al seguente lemma:
       \subsection*{Lemma}
        L'applicazione $\varsigma : \mathbb{R}^2 \rightarrow \mathbb{C}$ è un isomorfismo
        \begin{proof}
            La dimostrazione è banale basta considerare la definizione di $\mathbb{R}^2$.    
        \end{proof}
    Bisogna precisare però il fatto che $\mathbb{C}$ è isomorfo a $\mathbb{R}^2$ solo se li si considera come spazi vettoriali, infatti per costruire un isomorfismo tra algebre bisogna definire una nuova struttura come segue:
    \begin{definition}
        Il prodotto tra vettori è dato dall'operatore binario: 
        \begin{equation*}
            \xi : \mathbb{R}^2 \times \mathbb{R}^2 \rightarrow \mathbb{R}^2
        \end{equation*}
        tale che ad ogni coppia $((a,b),(c,d))$ associa il vettore $(ac - bd, bc + ad).$

    \end{definition}

    Consideriamo il numero complesso $z \in \mathbb{C}$ con $z = (a,b)$, con l'operazione appena definita possiamo perciò costruirlo come segue:
    \begin{equation*}
        a + ib = (a,0) + (0,1)(b,0) = (a, 0) + (0,b) = (a,b) 
    \end{equation*}
        Dotando perciò lo spazio vettoriale di questo prodotto si ottiene la capacità di dividere uno scalare per un vettore di $\mathbb{R}^2$. \\
        Infatti: 
        \begin{equation*}
            \frac{\alpha }{(\beta ,\gamma)} = \alpha (\delta ,\varepsilon) = (\alpha, 0)(\delta ,\varepsilon) = (\alpha \delta, \alpha \varepsilon )
        \end{equation*}

        Dove $\alpha,\beta,\gamma \in \mathbb{R}^2$ e $(\delta ,\varepsilon) = (\beta, \gamma)^{-1}$. \\
        Tale inverso esiste perchè definendo così il prodotto abbiamo reso $\mathbb{R}^2$ un campo.

        la definizione di derivata torna ad avere senso.
\end{document}